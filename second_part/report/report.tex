\documentclass{article}

\usepackage{amsmath}
\usepackage{graphicx}
\usepackage[italian]{babel}
\usepackage{animate}
\usepackage{verbatim}

\title{Relazione per il Secondo Progetto di Algoritmi e Strutture Dati e Laboratorio}
\date{12-09-2020}
\author{Alessandro Zanatta \\ 143154 \\ zanatta.alessandro@spes.uniud.it\\ \and Christian Abbondo \\ 144033 \\ abbondo.christian@spes.uniud.it\\ \\}

\begin{document}
	
	% Generate first page with no page numbers
	\pagenumbering{gobble}
	\maketitle
	\newpage
	
	
	% Generate table of contents (entries correspond to sections, subsections, ...)
	\pagenumbering{arabic}
	\tableofcontents	
	\newpage
	
	\section{Introduzione}
	La seguente relazione si propone di analizzare i tempi di esecuzione di operazione su tre diversi tipi di alberi binari: 
	\begin{itemize}
		\item Binary Search Tree (BST)
		\item Adelson-Velsky and Landis Tree (AVL)
		\item Red-Black Tree (RBT)
	\end{itemize}
	 
	Il linguaggio scelto e utilizzato per implementare le strutture dati e gli algoritmi, nonchè per il calcolo dei tempi di esecuzione, è C, in quanto è indubbiamente uno dei linguaggi più efficienti e veloci, al costo di una maggiore complessità del codice. \\ In questo modo i tempi di esecuzione ottenuti sono liberi da operazioni sulla memoria (come ad esempio un garbage collector) che vengono invece effettuate in momenti opportuni.
	\newpage
	
	
	\section{Considerazioni teoriche}
	In questa sezione si discuteranno le aspettative teoriche relative ai tempi di esecuzione degli alberi binari in questione.
	
	\subsection{Costo asintotico}
	Ricordiamo, innanzitutto, il costo asintotico delle operazioni al variare del numero di elementi $n$ presenti nell'albero:
	
	\paragraph{Binary Search Tree}\mbox{}\\
	Operazione di ricerca: $\Theta\left(n\right)$ nel caso peggiore, $O\left(\log{n}\right)$ nel caso medio.\\
	Operazione di inserimento: $\Theta\left(n\right)$ nel caso peggiore, $O\left(\log{n}\right)$ nel caso medio. \\
	
	\paragraph{Adelson-Velsky and Landis Tree}\mbox{}\\
	Operazione di ricerca: $O\left(\log{n}\right)$ in tutti i casi.\\
	Operazione di inserimento: $O\left(\log{n}\right)$ in tutti i casi. \\
	
	\paragraph{Red-Black Tree}\mbox{}\\
	Operazione di ricerca: $O\left(\log{n}\right)$ in tutti i casi.\\
	Operazione di inserimento: $O\left(\log{n}\right)$ in tutti i casi. \\
	
	\newpage
	\subsection{Aspettative teoriche}
	Si discutono di seguito le aspettative teoriche riguardanti i diversi tipi di alberi.
	
	\paragraph{BST}
	Data la maggior semplicità e i maggiori costi asintotici di questo tipo di albero, ci si aspetta una maggiore inefficienza, accentuata se l'albero si sbilancia. Per input casuali potrebbe comunque eseguire le operazioni in un tempo comparabile agli altri due alberi binari in quanto potrebbe rimanere, a causa della natura casuale delle chiavi dei nodi che vengono inseriti, relativamente bilanciato.\\
	Per input fortemente ordinati, questo albero sarà invece certamente molto inefficiente.
	
	\paragraph{RBT e AVL}	
	Sia i RBT che gli AVL impiegano tempo asintotico logaritmico per qualsiasi input e operazione, tuttavia è importante notare le differenze di questi due tipi di alberi.\\
	I RBT tendono a essere meno bilanciati rispetto agli alberi di tipo AVL. Formalmente, dato un nodo $x$ e denotato come $x.left$ e $x.right$ rispettivamente il figlio sinistro e destro di $x$, si ha che $\text{h}\left(x.left\right) \leq 2\cdot \text{h}\left(x.right\right)$ (e viceversa). Negli alberi AVL, invece, la differenza di altezza fra due sottoalberi di un certo nodo $x$ non supera mai l'unità. Ci si aspetta quindi un tempo di ricerca maggiore per i RBT rispetto agli AVL. \\
	Consideriamo inoltre la procedura di inserimento in un RBT. Questa richiede al più due cammini radice-foglia, in particolare ne richiede sempre uno per arrivare in una foglia (in cui verrà inserito il nodo con la chiava appropriata) e ne richiede al più un altro (o solo parte di esso) per ri-bilanciare l'albero. Per quanto riguarda gli AVL, invece, sono sempre richiesti esattaemente due cammini, il primo per l'inserimento e il secondo per ri-bilanciare l'albero, similmente al caso dei RBT. Per quanto riguarda l'operazione di inserimento, quindi, ci si aspetta che i RBT siano maggiormente efficienti degli alberi di tipo AVL.
	
	\newpage
	
	\section{Note implementative}
	\label{section:impl_notes}
	Si discutono ora alcune scelte implementative effettuate.
	
	\paragraph{Generazione di numeri pseudo-casuali}
	Dato che la funzione $\text{rand()}$ di C è piuttosto lenta e tende a fornire gli stessi interi dopo un numero relativamente piccolo di iterazioni, si utilizzerà una implementazione dell'algoritmo Mersenne Twister.
	
	\paragraph{Scomputazione dei tempi di inizializzazione}
	Si è ritenuto opportuno, al fine di ottenere dei tempi minormente affetti da errori casuali, scomputare il tempo di generazione di $n$ numeri casuali utilizzando l'algoritmo sopra nominato. Si discute il modo in cui si è effettuato ciò nella sezione successiva alla presente.
	
	\newpage
	\section{Calcolo del tempi di esecuzione}	
	Si riportano di seguito alcune	considerazioni riguardo il calcolo dei tempi effettuato nell'elaborato.
	
	\paragraph{Risoluzione}
	Si è scelto di utilizzare il clock offerto dalla funzione $\text{clock\_gettime}$. Il primo parametro della funzione indica il tipo di clock, mentre il secondo parametro è un puntatore ad una struct contentente il tempo all'istante della chiamata a funzione. Il tipo di clock utilizzato ($\text{CLOCK\_MONOTONIC\_RAW}$) è, come suggerito dal nome stesso, di tipo monotonico. \\ Per ottenere una misurazione della risoluzione migliore si è utilizzata la risoluzione mediana, calcolata su un campione di 10000 elementi.
	
	\paragraph{Scomputazione dei tempi di inizializzazione}
	Al fine di scomputare correttamente dai tempi di esecuzione delle operazioni il tempo di generazione di $n$ numeri casuali, si è utilizzata una funzione apposita che calcola il tempo mediano di tale operazione. Al fine di ottenere un errore relativo minore di $0.5\%$ si è ripetuta l'inizializzazione sufficienti volte affinché la seguente equazione fosse soddisfatta:
	
	\[
		\text{diff}\left(end, start\right) > resolution \cdot \left(1+\frac{1}{0.005}\right)  
	\]
	
	dove $\text{diff}$ indica l'intervallo temporale trascorso fra $start$ e $end$ e $resolution$ è la risoluzione del clock utilizzato. \\ Questo procedimento è stato poi iterato un certo numero di volte e dal campione di dati ottenuto si è estratta la mediana.
	
	\paragraph{Tempo di esecuzione degli algoritmi}
	Per ottenere misurazioni con errore relativo minore di $0.5\%$ si è operato nel seguente modo:
	
	\begin{itemize}
		\item Calcolo del tempo di generazione dei numeri casuali secondo la procedura descritta nel paragrafo precedente
		\item Calcolo del tempo di esecuzione delle operazioni. Si itera questo passaggio finché non è soddisfatta la seguente equazione:
		
			\[
				\text{diff}\left(end, start\right) > initialization + resolution \cdot \left(1+\frac{1}{0.005}\right)  
			\]
			
			dove $initialization$ è il tempo calcolato al primo punto.
	\end{itemize}
	
	% the paragraph under this does not fit in the current page :/
	\newpage
	
	\paragraph{Errore relativo massimo totale}
	Calcolando i tempi nei modi indicati negli ultimi due paragrafi si è ottenuto un errore relativo complessivo $\leq1\%$. L'errore relativo totale è, infatti, dato dalla somma degli errori. Essendoci quindi due misurazioni affette da errore\footnote{La prima è quella relativa al calcolo del tempo di inizializzazione, la seconda è quella relativa al tempo di esecuzione dell'algoritmo.}, entrambe con errore relativo pari o minore a $0.5\%$, l'errore relativo massimo totale è effettivamente dell'$1\%$.
	
	\paragraph{Problematiche riscontrate}
	Durante le prime misurazioni, si era notata una deviazione standard piuttosto elevata, indice di una grande variabilità nelle misurazioni ottenute. Al fine di ottenere delle misurazioni migliori, cioè meno sensibili ad outliers e rumore, si è scelto di utilizzare, al posto della media, la mediana dei tempi di esecuzione e, al posto della deviazione standard, la deviazione mediana assoluta, definita come $\textrm{MAD}=\textrm{median}\left(\mathopen|X_{i}-\textrm{median}\left(X\right)\mathclose|\right)$. Questi indici, in quanto indici di posizione, sono più robusti della media e della deviazione standard e hanno permesso di ottenere dei tempi sperimentali più accurati e minormente affetti da errori casuali.
	
	\newpage	
	\section{Analisi degli tempi di esecuzione}
	
	
	\newpage

	\section{Conclusioni}
	
	
\end{document}
